% Chapter 1

\chapter{Introduction} % Main chapter title

\label{Chapter1} % For referencing the chapter elsewhere, use \ref{Chapter1} 

%----------------------------------------------------------------------------------------

% Define some commands to keep the formatting separated from the content 
\newcommand{\keyword}[1]{\textbf{#1}}
\newcommand{\tabhead}[1]{\textbf{#1}}
\newcommand{\code}[1]{\texttt{#1}}
\newcommand{\file}[1]{\texttt{\bfseries#1}}
\newcommand{\option}[1]{\texttt{\itshape#1}}

%----------------------------------------------------------------------------------------

\section{Parental controls}

\subsection{Overview}
Parental controls developed in the digital era as a means to allow parents to restrict the access of content to their children and may be included in digital television services, computer and video games and mobile devices. The content may not be appropriate for their age and is aimed more at adult audiences. The characteristics of inappropriate content depends for each parent, and is also correlated with the child's age and maturity level and includes information and images that can upset the child, inaccurate information or information that can cause dangerous behavior. Some of this content could be:

\begin{itemize}
\item pornographic material
\item content containing swearing
\item sites that encourage vandalism
\item pictures, videos or games which shows images of violence
\item gambling sites
\item unmoderated chatrooms
\end{itemize}

It is very easy for the child to stumble upon unsuitable sites by accident on any internet enabled device, like mobile phone or tablet and it can be difficult to monitor and filter the content. \parencite{innapropriateContent}

Parental control solutions fall into four categories:

\begin{itemize}
\item content filters, which limit access to different types of inappropriate content
\item usage control, which works by constraining the usage of certain devices by placing time-limits on usage or forbid some types of usage
\item computer usage management tools, which enforces the use of certain software
\item monitoring, which can track the activity when using the devices
\end{itemize}

The rising availability of the Internet increased the demand for methods of parental control that restrict content. Mobile phones offer the most convenient and constant method for content access, and teens ages 13 to 17 are going online frequently. A study by Pew Research Center found that 92\% of teens report going online daily, 24\% of which are using the internet almost constantly, 56\% going online several times a day and 12\% reporting once a day use. Only 6\% go online weekly and 2\% less often. \parencite{lenhart2015teens}

\begin{figure}[th]
\centering
\includegraphics{Figures/frequency-of-internet-use-by-teens}
\decoRule
\caption[Frequency of Internet Use by Teens]{Frequency of Internet Use by Teens}
\label{fig:frequency-of-internet-use-by-teens}
\end{figure}

The same study finds that nearly three-quarters have or have access to a smartphone and only 30\% have a basic phone and 12\% of teens 13 to 17 have no cell phone of any type.

\subsection{Techniques}

\subsection{Content filters}

The increased use of mobile devices has created a demand for parental controls for these devices. The first carrier which offered age-appropriate content filters was Verizon, in 2007. With the release of iPhone OS 3.0 in 2009, Apple introduced a mechanism to create age brackets for users, to block unwanted applications from being downloaded.

%----------------------------------------------------------------------------------------

\section{A self regulation approach}

%----------------------------------------------------------------------------------------
